\documentclass[../principal.tex]{subfiles}

\begin{document}

  \thispagestyle{empty}
  \begin{center}
    \textbf{Resumen}
  \end{center}

  En el presente proyecto se desarrolla un sistema que cumple dos funciones, la primera es la de controlar la apertura de puertas mediante un subsistema que recoge los datos que envían los sensores de huellas dactilares, mismos que son procesados para verificar los permisos de un usuario para acceder por una de las puertas. La segunda función es la de hacer un puente interfaz entre el sensor de huellas dactilares y el servidor MQTT, este último será el que envíe las órdenes para realizar alguna acción en los sensores como por ejemplo la de grabar una huella en una posición definida de la base de datos.

  En este proyecto también se desarrollan los programas necesarios para ambos usos del hardware obtenido, a su vez se desarrollan las bases de datos que se utilizan para las secuencias de acciones del sistema, así como también para almacenar los datos de los usuarios y los componentes del sistema.

  Se analizan los conceptos fundamentales de seguridad física de ambientes e infraestructura que se contrastan con normas nacionales e internacionales. Por tal motivo también se realiza un estudio de las vulnerabilidades del sistema desarrollado y los pasos a seguir para evitar ataques o fallas en el sistema.

  Por último también se realiza un estudio de costos y una comparación del sistema propuesto frente a un sistema de características similares que puede encontrarse en el mercado, para concluir con la realización de una instalación piloto en el centro de datos de la Agencia para el Desarrollo de la Sociedad de la Información en Bolivia.

  \textbf{Palabras claves: } Biométrico, Puertas, Accesos, Sensores, MQTT, Arduino, PostgreSQL, Node

  \newpage
  \thispagestyle{empty}
  \begin{center}
    \textbf{Abstract}
  \end{center}

    The present project performs the development of a system fulfills two functions, the first is to control the opening of doors by means of a subsystem that collects the data sent by the fingerprint sensors, which are processed to verify the permissions of a user to access by one of the doors. The second function is to make an interface bridge between the fingerprint sensor and the MQTT server, this latter will be the one that sends the commands to perform some action on the sensors such as for example recording a fingerprint in a defined position of the database.

    This project also develops the necessary programs for both functions of the obtained hardware, in turn develop the databases that are used for the sequences of actions of the system, as well as for storing users data and system components.

    It will be analyze the most important concepts of physical security of environments and infrastructure that are contrasted with national and international standards. For this reason also It will be performs a study of the vulnerabilities of the system developed and the steps to follow to avoid attacks or failures in this system.

    Finally, will be performs a study of costs and a comparison of the proposed system against a system of similar characteristics that can be found in the market, in order to conclude with the realization of a pilot installation in the data center of the Agency for the Development of The Information Society in Bolivia.

  \textbf{Keywords: } Biometric, Doors, Access, Sensors, MQTT, Arduino, PostgreSQL, Node

\end{document}
